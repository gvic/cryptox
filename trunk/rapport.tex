\documentclass[a4paper, 11pt]{article}

\usepackage[francais]{babel}
\usepackage[utf8]{inputenc}
\usepackage[T1]{fontenc}
\usepackage{listings}

\lstset{language={Python},}


\begin{document}

\title{Rapport du projet de RO}
\author{Paul Bonaud\\
  Victor Godayer}
\date\today

\maketitle

\begin{abstract}
  Ce projet implante une solution de décryptage face à un chiffre
  polyalphabétique périodique .

  Ce décryptage se réalise selon plusieurs étapes que nous décrirons
  dans le présent rapport.
\end{abstract}
\newpage
\tableofcontents
\newpage


\section{Introduction}
Notre travail se compartimente en 3 parties, selon le sujet fourni.

\paragraph{}
Dans un premier temps nous avons aborder le problème de la période.
Trouver la période revient à trouver le nombre d'alphabets utilisés pour
coder le texte clair.

\paragraph{}
Dans un second temps, il nous est nécessaire de rechercher des cribles
dans le chiffre, ceci nous permettant d'établir un lien entre les
lettres du crible et les alphabets.

La qualité de la recherche du crible s'appuie essentiellement sur
trois \textit{indicateurs} : la \textit{pertinence}, la
\textit{rareté} et la \textit{vraissemblance}.

\paragraph{}
Enfin il nous était demandé d'avoir une approche de programmation génétique pour
pouvoir retrouver complètement les alphabets servant à chiffrer le
texte clair. Cependant nous n'avons pas eu le temps de nous consacrer à
cette tâche.


\section{Conception}

\subsection{Choix du langage}
Nous connaissions le langage \textit{Python}.
Ce dernier étant partiulièrement bien adapté pour la manipulation
du type de donnée \textit{string}, nous l'avons choisi pour implanter
le projet.
Appartenant à la catégorie des langages fonctionnels, c'est également
un aspect pratique pour le projet.

\subsection{Détermination de la période}

\subsubsection{Méthode courte}
Nous avons implanté la méthode de Kasiski courte de la manière
suivante.

Pour un texte chiffré et une taille \textit{l} donnée la fonction recherche toutes
les sous-chaines de la taille \textit{l} qui se répetent dans le texte
chiffré.

Une fonction de test se charge de tester toutes les valeurs de l
intéressantes.
Pour voir les résultats des tests se reporter à la partie test.

\paragraph{}
Voici le code python (très accessible) de cette fonction:
\newpage
\begin{lstlisting}
def kasiski_court(text, l):
    for i in range(len(text)-l):
        target = text[i:i+l]
        found = text[i+l:].find(target)
        if found != -1:
            f = found+i+l
            if i>0 and text[i-1:i+l] == text[f-1:f+l]:
                continue
            if i+l < len(text) and text[i:i+l+1] == text[f:f+l+1]:
                continue            

            print '\%-10s \%3d' \% (target, found+l)
\end{lstlisting}


\subsubsection{Méthode longue}
... 

\subsection{Recherche de crible}

Comme nous n'avons pas pu mener à bout cette partie, la problématique
de la présentation du résultat n'a même pas eu être aborder.
Voici les \textit{recherches} et codages que nous avons effectué sur
les notions de pertinence et de rareté.

\subsubsection{Pertinence}
Même après avoir réfléchis longuement sur cette notion nous n'avons
pas pu établir un indicateur réellement fiable pour un crible donné.
Nous n'avons pas réussi à trouver des règles claires nous permettant
d'évaluer la pertinence du crible en fonction d'une période.

Comme un exemple vaut mieux qu'un long discours, se reporter à la
partie test pour voir les résultats de la fonction pertinence.


\subsubsection{Rareté}
Pour la rareté du crible nous avons, pour une table de
fréquence donnée, calculé une valeur entre 0 et 1, 0 représentant une
rareté nulle et 1 une rareté maximale.

La rareté du crible se fait simplement en sommant les fréquences
d'apparition de chaque lettre puis en divisant cette somme par le
nombre de lettre multiplié par la fréquence maximale pour la table des
fréquences donnée.

C'est une sorte de moyenne des fréquences des lettres qui constituent
le mot.

\subsubsection{Vraissemblance}
Nous n'avons pas réussi à établir une fonction d'évaluation de la
vraisemblance d'une position d'un crible.

Cependant nous avons codé une fonction \textit{table\_frequences} qui, à un
crible et une période données, nous renvoie une liste de la longueur
de la \textit{période}.

On nommera cette liste \textit{l}.

Chaque élément de \texit{l}, est une liste des
lettres qui se trouvent à une position \textit{j} dans le crible, modulo
la valeur de la \textit{période}.

Nous appelerons cette seconde liste \textit{l$'$}

Chaque \textit{l'} listent des couples contenant la lettre et son
occurrence. L'occurrence est en le nombre de fois ou on a trouvé cette
lettre à la position \textit{j} modulo la \textit{période}

Voir dans la partie test pour avoir des exemples clairs.



\subsection{Reconstitution du chiffre}

\begin{itemize}
\item Quelle est votre définition du volume mémoriel ?
\item Quelle est votre définition de l'ADN ?
\item Comment se déroulent les croisements ?
\item Qu'avez vous testé ?
\end{itemize}


\subsubsection{Volume mémoriel}
\subsubsection{Programmation génétique}
\begin{itemize}
\item ADN et croisements
\item Séléction artificielle
\item Conclusion sur la partie 3 (a mettre dans tests peut etre ?)
\end{itemize}



\section{Tests}

\subsection{Détermination de la période}

\subsubsection{Méthode courte}

Les longueurs possible des sous chaines sont automatiquement testées
dans le programme de test.
Celles-ci vont de la longueur du texte divisé par 2 à 2.

\begin{lstlisting}
\$ python kasiski_court.py
text? -> texte exemple de la jangada
...
...
test d'une longueur de sous chaine de:5
test d'une longueur de sous chaine de:4
DDQF       186 # distance entre les sous chaines repetees
KYUU        12
test d'une longueur de sous chaine de:3
RYM        192
TOZ        186
RPL         60
HHH         54

determination de la période: pgcd de [186, 12, 192, 186, 60, 54] =
6
\end{lstlisting}

\subsubsection{Méthode longue}
...

\subsection{Crible}

\subsubsection{Pertinence}
Ces tests ne tiennent pas forcément compte du fait que le crible est
un mot avec du sens.
Pour mettre à rude épreuve la logique de la fonction il était plus
simple d'employer une séquence de lettres avec des répétitions aux
endroits qui nous intéressait.

\begin{lstlisting}
>> p = 6 # la periode
>> pertinence('soldats',p)
>> 1.0
>> pertinence('soldatso',p)
>> 1.0 # Ici on pourrait s'attendre à un indicateur plus eleve
\end{lstlisting}

\subsubsection{Rareté}
Voici sous forme d'un tableau la rareté obtenu pour différent crible
de tests : \\

\begin{tabular}{|c|c|c|c|c|}
  \hline
   ``soldats'' & ``escorte'' & ``zebre'' & ``legerement'' & ``wagon'' \\
  \hline
   0.535415100974 & 0.406934585165 & 0.49089956356 & 0.38087729338 & 0.692843895006  \\
  \hline
\end{tabular}

\subsubsection{Vraisemblance}

Le premier paramètre de la fonction est le crible et le seconde la période

\begin{lstlisting}
>>> from crible import *
>>> table_frequences("soldats",6)
[('s', 2), ('o', 1), ('l', 1), ('d', 1), ('a', 1), ('t', 1)]

>>> table_frequences("onomatope",6)
[('o', 2), ('p', 1, 'n', 1), ('e', 1, 'o', 1), ('m', 1), ('a', 1),
  ('t', 1)]

>>> table_frequences("tagadaga",2)
[('t', 1, 'g', 2, 'd', 1), ('a', 4)]
\end{lstlisting}


\subsection{Reconstitution du chiffre}


\section{Conclusion}



\end{document}
