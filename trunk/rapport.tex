\documentclass[a4paper, 11pt]{article}

\usepackage[francais]{babel}
\usepackage[utf8]{inputenc}
\usepackage[T1]{fontenc}



\begin{document}

\title{Rapport du projet de RO}
\author{Paul Bonaud\\
Victor Godayer}
\date\today

\maketitle

\begin{abstract}
  Ce projet implante une solution de décryptage face à un chiffre
  polyalphabétique périodique .

  Ce décryptage se réalise selon plusieurs étapes que nous décrirons
  dans le présent rapport.
\end{abstract}
\newpage
\tableofcontents
\newpage


\section{Introduction}
Notre travail se compartimente en 3 parties, selon le sujet fourni.
Dans un premier temps nous avons aborder le problème de la période

\section{Conception}

Questions auxquelles il faudra répondre dans ce rapport et durant le
test : 
\begin{itemize}
  \item Partie 1:
    \begin{itemize}
      \item Avez vous traité la méthode courte ?
      \item Comment ?
      \item A quels tests avez-vous procédé ?
    \end{itemize}
  \item Partie 2:
    \begin{itemize}
      \item Comment présentez vous le résultat d'une recherche ?
    \end{itemize}
  \item Partie 3:
    \begin{itemize}
      \item Quelle est votre définition du volume mémoriel ?
      \item Quelle est votre définition de l'ADN ?
      \item Comment se déroulent les croisements ?
      \item Qu'avez vous testé ?
    \end{itemize}

\end{itemize}

Notre


\subsection{Choix du langage}
Nous connaissions le langage \textit{Python}
Ce dernier étant partiulièrement bien adapté pour la manipulation
du type de donnée \textit{string}, nous l'avons choisi pour implanter
le projet.

\subsection{Détermination de la période}
\subsubsection{Méthode courte}
Nous avons implanté la méthode de Kasiski courte de la manière
suivante.

Pour un texte chiffré et une taille \textit{l} donnée la fonction recherche toutes
les sous-chaines de la taille \textit{l} qui se répetent dans le texte chiffré.
\subsubsection{Méthode longue}


\section{Tests}

\section{Conclusion}



\end{document}
