\documentclass[a4paper, 11pt]{article}

\usepackage[francais]{babel}
\usepackage[utf8]{inputenc}
\usepackage[T1]{fontenc}

\begin{document}

\title{Rapport du projet de RO}
\author{Paul Bonaud\\
  Victor Godayer}
\date\today

\maketitle

\begin{abstract}
  Ce projet implante une solution de décryptage face à un chiffre
  polyalphabétique périodique .

  Ce décryptage se réalise selon plusieurs étapes que nous décrirons
  dans le présent rapport.
\end{abstract}
\newpage
\tableofcontents
\newpage


\section{Introduction}
Notre travail se compartimente en 3 parties, selon le sujet fourni.

\paragraph{}
Dans un premier temps nous avons aborder le problème de la période.
Trouver la période revient à trouver le nombre d'alphabets utilisés pour
coder le texte clair.

\paragraph{}
Dans un second temps, il nous est nécessaire de rechercher des cribles
qui permettent de ... ?

La qualité de la recherche du crible s'appuie essentiellement sur
trois \textit{indicateurs} : la \textit{pertinence}, la
\textit{rareté} et \textit{vraissemblance}.

\paragraph{}
Enfin il était nécessaire une approche de programmation génétique pour
pouvoir retrouver complètement les alphabets servant a chiffrer le
texte clair.


\section{Conception}

\subsection{Choix du langage}
Nous connaissions le langage \textit{Python}
Ce dernier étant partiulièrement bien adapté pour la manipulation
du type de donnée \textit{string}, nous l'avons choisi pour implanter
le projet.

\subsection{Détermination de la période}

\begin{itemize}
\item Avez vous traité la méthode courte ?
\item Comment ?
\item A quels tests avez-vous procédé ?
\end{itemize}


\subsubsection{Méthode courte}
Nous avons implanté la méthode de Kasiski courte de la manière
suivante.

Pour un texte chiffré et une taille \textit{l} donnée la fonction recherche toutes
les sous-chaines de la taille \textit{l} qui se répetent dans le texte
chiffré.

\subsubsection{Méthode longue}
... 

\subsection{Recherche de crible}
\begin{itemize}
\item Comment présentez vous le résultat d'une recherche ?
\end{itemize}
\subsubsection{Pertinence}
...
\subsubsection{Rareté}
Pour la rareté du crible nous avons simplement pour une table de
fréquence donnée, calculé une valeur entre 0 et 1, 0 représentant une
rareté nulle et 1 une rareté maximale.

La rareté du crible se fait simplement en sommant la fréquence
d'apparition de chaque lettre puis en divisant cette somme par le
nombre de lettre multiplié par la fréquence maximale.

C'est une espèce de moyenne des fréquences des lettres qui constituent
le mot.
\subsubsection{Vraissemblance}
...

\subsection{Reconstitution du chiffre}

\subsubsection{Volume mémoriel}
\subsubsection{Programmation génétique}
\begin{itemize}
\item ADN et croisements
\item Séléction artificielle
\item Conclusion sur la partie 3 (a mettre dans tests peut etre ?)
\end{itemize}

\begin{itemize}
\item Quelle est votre définition du volume mémoriel ?
\item Quelle est votre définition de l'ADN ?
\item Comment se déroulent les croisements ?
\item Qu'avez vous testé ?
\end{itemize}


\section{Tests}

\section{Conclusion}



\end{document}
